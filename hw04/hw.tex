\documentclass{article}
\usepackage{setspace}
\usepackage{geometry}
\usepackage[utf8]{inputenc}
\usepackage{amsmath,amsthm,amssymb}
\usepackage{mathtools}

\geometry{letterpaper, portrait, margin=1in}
\setstretch{1.5}
\title{Homework 4}
%\date{1-18-2020}
\author{Runmin Lu}

\begin{document}
	\maketitle
	%\newpage
	
	\section*{Exercise 2.4}
	\subsection*{(a)}
		Construct $d+1$ points as follows and put all of them into columns of a matrix called $X$ where $X$ has all 1's on the diagonal and the first row. The rest of the entries are 0's. It looks like
		\begin{align*}
			\begin{pmatrix}
				1 & 1 & 1 & ... & 1 & 1\\
				0 & 1 & 0 & ... & 0 & 0\\
				0 & 0 & 1 & ... & 0 & 0\\
				...\\
				0 & 0 & 0 & ... & 0 & 1
			\end{pmatrix}
		\end{align*}
		Clearly it's upper triangular so it must be singular, which means that it spans the entire $\mathbb R^{d+1}$. Therefore $\mathbf wX$ can span $\mathbb R^{d+1}$, so each entry can take positive or negative values, and hereby implements all $2^{d+1}$ dichotomies.
	
	\subsection*{(b)}
		As the hint indicates, the span of any $d+1$ by $d+2$ matrix is at most $d+1$, which means that one entry of $\mathbf wX$ (which has dimension $d+2$) is linearly dependent on all other $d+1$ entries. For every combination of those $d+1$ entries, the one dependent entry must only take one value. Therefore not all $2^{d+2}$ dichotomies can be implemented.
		
	\section*{Problem 2.3}
	\subsection*{(a)}
		$N$ data points split the number line into $N-1$ inner intervals and 2 outer intervals.\\
		For each of the $N-1$ intervals we have 2 dichotomies by shooting the ray in 2 directions.\\
		For each of the 2 intervals, we have 2 dichotomies: all $+1$'s and all $-1$'s.\\
		Therefore $m_{\mathcal H} (N) = 2(N-1) + 2 = \boxed{2N}$\\
		$d_{VC} = \boxed{2}$ because for 3 data points we can't have $+1$’s on two ends and $-1$’s in the middle.\\
		2 data points can be shattered because $2(2) = 2^2$
		
	\subsection*{(b)}
		$N$ data points split the number line into $N-1$ inner intervals and 2 outer intervals.\\
		The boundaries are either in the same interval or different.\\
		If they're in the same interval, then there are only 2 dichotomies: all $+1$'s or all $-1$'s.\\
		Otherwise, if the boundaries are in the $N-1$ intervals, then we have $2{N-1 \choose 2}$ dichotomies, where each section contains 1 or more $+1$'s or $-1$'s.\\
		If we fix one boundary in an outer interval, then there are $2(N-1)$ dichotomies with 1 or more $+1$'s followed by 1 or more $-1$'s or the other way around.\\
		In total,
		\begin{align*}
			m_{\mathcal H}(N) &= 2 + 2{N-1 \choose 2} + 2(N-1)\\
			&= 2 + 2\frac{(N-1)(N-2)}2 + 2N - 2\\
			&= (N-1)(N-2) + 2N\\
			&= \boxed{N^2 - N + 2}
		\end{align*}
		$d_{VC} = \boxed 3$ because we can't have 4 data points ordered as $-1, +1, -1, +1$ but $N = 3$ is shattered because $2^3 = 8 = 3^2 - 3 + 2$
		
	\subsection*{(c)}
		Because the only property that matters to a data point is its distance to the origin, such $\mathcal H$ in $\mathbb R^d$ is simply equivalent to a positive interval in $\mathbb R^+$. Therefore, $m_{\mathcal H}(N) = \boxed{\frac12 N^2 + \frac12 N + 1}$ and $d_{VC} = \boxed 2$ as shown in the example from last homework.
		
	\section*{Problem 2.8}
	$1+N$ is possible because it's for Positive Rays.\\\\
	$1 + N + \frac{N(N-1)}2$ is possible because it's for Positive Intervals.\\\\
	$2^N$ is possible because it's for Convex Sets.\\\\
	$2^{\lfloor \sqrt N \rfloor}$ is not possible.
	\begin{proof}\ \\
	Assume for contradiction that it is possible. Then
		\begin{align*}
			m_{\mathcal H}(1) &= 2^1\\
			m_{\mathcal H}(2) &= 2^1 < 2 ^ 2\\
			d_{VC} &= 1\\
			m_{\mathcal H}(N) &\leq N+1\\
			\text{However }m_{\mathcal H}(25) &= 2^5 = 32 > 25 + 1\\
			&\bot
		\end{align*}
	\end{proof}\ \\
	$2^{\lfloor N/2 \rfloor}$ is not possible. Proof is similar to the one above: $d_{VC} = 1$ but $m_{\mathcal H}(6) = 8 > 6+1$.\\\\	
	$1 + N + \frac{N(N-1)(N-2)}6$ is not possible. Proof is again similar: $d_{VC} = 1$ but $m_{\mathcal H}(3) = 5 > 3 + 1$.
	
	\section*{Problem 2.10}
		Any $2N$ data points can be split into 2 sets of $N$ data points, each of whose maximum number of dichotomies is $m_{\mathcal H}(N)$.  Since $m_{\mathcal H}(2N)$ represents the maximum number of dichotomies of a combination of these 2 sets of $N$ data points, it therefore can be at most $m_{\mathcal H}(N)^2$.
		
	\section*{Problem 2.12}
		\begin{align*}
			\varepsilon &= \sqrt{\frac8N \ln \frac{4(2N)^{d_{VC}}}\delta}\\
			\varepsilon^2 &= \frac8N \ln \frac{4(2N)^{d_{VC}}}\delta\\
			N &= \frac8{\varepsilon^2} \ln \frac{4(2N)^{d_{VC}}}\delta\\
			&= \frac8{\varepsilon^2}d_{VC}\ln(2N) + \frac8{\varepsilon^2}\ln\frac4\delta\\
			\text{Let }\ln(2N) &\approx 10\\
			N &\approx  \frac8{0.05^2}10(10) + \frac8{0.05^2}\ln\frac4{0.05}\\
			&\approx \boxed{3.34 \times 10^5}
		\end{align*}
\end{document}