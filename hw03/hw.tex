\documentclass{article}
\usepackage{setspace}
\usepackage{geometry}
\usepackage[utf8]{inputenc}
\usepackage{amsmath,amsthm,amssymb}
\usepackage{mathtools}

\geometry{letterpaper, portrait, margin=1in}
\setstretch{1.5}
\title{Homework 3}
%\date{1-18-2020}
\author{Runmin Lu}

\begin{document}
	\maketitle
	%\newpage
	
	\section*{Exercise 1.13}
	\subsection*{(a)}
		2 cases where $h$ makes an error:\\
		$h = f, f \neq y$\\
		or\\
		$h \neq f, f = y$
		\begin{align*}
			P(h = f, f \neq y) + P(h \neq f, f = y) &= (1 - \mu)(1 - \lambda) + \mu\lambda\\
			&= 1 - \mu - \lambda + \mu\lambda + \mu\lambda\\
			&= \boxed{1 - \mu - \lambda + 2\mu\lambda}
		\end{align*}
	\subsection*{(b)}
	For some value of $\lambda$, the $\mu$ in the expression above will cancel out.
	\begin{align*}
		1 - \mu - \lambda + 2\mu\lambda &= 1 - \lambda + \mu(2\lambda - 1)\\
		2\lambda - 1 &= 0\\
		\lambda &= \boxed{\frac12}
	\end{align*}
	
	\section*{Exercise 2.1}
	\subsection*{1}
		$k = 2$ because for 2 points, you cannot have the left being $+1$ and the right being $-1$.\\
		$m_H(2) = 2+1 = 3 < 2^2 = 4$
		
	\subsection*{2}
		$k = 3$ because you cannot have $+1$'s on two ends and $-1$'s in the middle.\\
		$m_H(3) = {4 \choose 2} + 1 = 6 + 1 = 7 < 2^3 = 8$
	\subsection*{3}
		Break point does not exist.
		
	\section*{Exercise 2.2}
	\subsection*{(a)}
	\textbf{(i)}
		\begin{align*}
			\text{LHS: } m_H(N) &= N + 1\\
			\text{RHS: } \sum\limits_{i = 0}^1{N\choose i} &= {N \choose 0} + {N \choose 1}\\
			&= 1 + N\\
			\text{LHS} &\leq \text{RHS}
		\end{align*}
	\textbf{(ii)}
		\begin{align*}
			\text{LHS: } m_H(N) &= \frac12N^2 + \frac12N + 1\\
			\text{RHS: } \sum\limits_{i = 0}^2{N\choose i} &= {N \choose 0} + {N \choose 1} + {N \choose 2}\\
			&= 1 + N + \frac{N(N-1)}2\\
			&= 1 + N + \frac{N^2}2 - \frac N2\\
			&= 1 + \frac N2 + \frac{N^2}2\\
			\text{LHS} &\leq \text{RHS}
		\end{align*}
	\textbf{(iii)}\\
		Break point does not exist.
		
	\subsection*{(b)}
		No.
		\begin{proof}\ \\
			Assume there exists such a hypothesis set. Then
			\begin{align*}
				m_H(1) &= 1 + 2^0 = 2 = 2^1\\
				m_H(2) &= 2 + 2^1 = 4 = 2^2\\
				m_H(3) &= 3 + 2^1 = 5 < 2^3				
			\end{align*}
			We found a break point $k = 3$.
			\begin{align*}
				m_H(N) &= \sum\limits_{i = 0}^2{N\choose i}\\
				&= 1 + \frac N2 + \frac{N^2}2\\
				&\in O(N^2)\
			\end{align*}
			However, we're given that $m_H(N) = N + 2^{\lfloor N/2 \rfloor} \in \Omega(2^{N/2})$. Contradiction.\\
			Therefore no such hypothesis set exists.
		\end{proof}
		
	\section*{Exercise 2.3}
	\subsection{(i)}
		\begin{align*}
			d_{VC} &= k - 1\\
			&= 2 - 1\\
			&= \boxed{1}
		\end{align*}
	\subsection{(ii)}
		\begin{align*}
			d_{VC} &= k - 1\\
			&= 3 - 1\\
			&= \boxed{2}
		\end{align*}
	\subsection*{(iii)}
		\begin{align*}
			m_H(N) &= 2^N\\
			d_{VC} &= \boxed{\infty}
		\end{align*}
		
	\section*{Exercise 2.6}
	\subsection*{(a)}
		$E_{test}(g)$ has the higher error bar because $\varepsilon \in O(\sqrt{\frac{\ln |H|}N})$ where $N$ is on the denominator. In this case the sample size for testing is smaller, which results in bigger $\varepsilon$.
	\subsection*{(b)}
		We want $E_{in}(g) \approx 0$, which means that we want to make the error bar for $E_{in}$ as small as possible. Therefore, we want to reserve more data used in selecting $g$ rather than testing.
	
	\section*{Problem 1.11}
		Let $e(g(x_i), y_i)$ denote the point wise error represented in the matrix.\\
		For the supermarket case:
		\begin{align*}
			e(g(x_i), y_i) &=
			\begin{cases}
				0 &g(x_i) = y_i\\
				1 &g(x_i) = +1, y_i = -1\\
				10 &g(x_i) = -1, y_i = +1
			\end{cases}
		\end{align*}supermarket
		For the CIA case:
		\begin{align*}
			e(g(x_i), y_i) &=
			\begin{cases}
				0 &g(x_i) = y_i\\
				1 &g(x_i) = -1, y_i = +1\\
				1000 &g(x_i) = +1, y_i = -1
			\end{cases}
		\end{align*}
		In general, for both cases with a sample size of $N$:
		\begin{align*}
			E_{in}(g) = \frac1N \sum\limits_{i=1}^N e(g(x_i), y_i)
		\end{align*}
		
	\section*{Problem 1.12}
	\subsection*{(a)}
		\begin{align*}
			E_{in}(h) &= \sum\limits_{n=1}^N(h - y_n)^2\\
			&= \sum\limits_{n=1}^N(h^2 - 2hy_n + y_n^2)\\
			&= Nh^2 - 2h\sum\limits_{n=1}^Ny_n + \sum\limits_{n=1}^Ny_n^2\\
			E_{in}'(h) &= 2Nh - 2 \sum\limits_{n=1}^Ny_n = 0\\
			h &= \frac1N \sum\limits_{n=1}^Ny_n
		\end{align*}
	\subsection*{(b)}
		Define a cutoff point $M$ where $\forall n \leq M: h \geq y_n$ and $\forall n > M: h < y_n$
		\begin{align*}
			E_{in}(h) &= \sum\limits_{n=1}^N|h - y_n|\\
			&= \sum\limits_{n=1}^M (h - y_n) + \sum\limits_{n=M+1}^N (y_n - h)\\
			&= Mh - \sum\limits_{n=1}^M y_n + \sum\limits_{n=M+1}^N y_n - (N-M)h\\
			&= (2M - N)h - \sum\limits_{n=1}^M y_n + \sum\limits_{n=M+1}^N y_n\\
			E_{in}'(h) &= 2M - N = 0\\
			M &= \frac N2 \implies h \text{ is median}
		\end{align*}
	\subsection*{(c)}
		$h_{\text{mean}}$ will increase because the mean is affected by every data point.\\
		$h_{\text{med}}$ won't change because the median is not affected by any outlier.
\end{document}